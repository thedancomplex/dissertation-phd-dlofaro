Feedback filtering allows us to removes the requirement that we know the joint's current position.
Similar to Equation~\ref{eq:reffiltermode} this method sets $\theta_c$ based on a filter length $L$ and the current desired value $\theta_r$.
However instead of assuming that we know all past $\theta_r$ we use the actual position $\theta_a$.
This method add compliance in a similar way to that of Section~\ref{sec:singlejointRefComplience}.


\begin{equation}\label{eq:refencmode}
\theta_c(N) = \frac{\theta_a(N)\cdot\left(L-1\right) + \theta_r(N)}{L}
\end{equation}

This causes three major effects: 

\noindent \textbf{Effect 1:} The movement of the joint is guaranteed to be filtered even if the previous reference is unknown.

\noindent \textbf{Effect 2:} The steady state error of the feedback filtering method $\theta_e^{fbfilter}$ is greater than that of the PID error $\theta_e$ in the direction of the moment acting on the joint.

\begin{equation}
\theta_e^{fbfilter} > \theta_e
\end{equation}

\noindent \textbf{Effect 3:} The joint's compliance has increased due to the effect of the moment applied to the joint has on the steady state error.

\begin{figure}
\centering

\begin{tikzpicture}[->,>=stealth',shorten >=1pt,auto,node distance=5cm,
  thick,main node/.style={fill=white!20,draw,font=\sffamily\Large\bfseries}]


  \node[main node] (ref) {Reference};
  \node[main node] (filter) [right=3.0cm of ref] {Filter};
  \node[main node] (hubo-ach) [below=1.0cm of filter] {Hubo-Ach};
  \node[main node] (hubo) [right=3.0cm of hubo-ach] {Hubo};




  \path[<->,dashed, every node/.style={font=\sffamily\small}]
    (hubo) edge node [above] {CAN} (hubo-ach);

  \path[->,every node/.style={font=\sffamily\small}]
    (ref) edge node [above] {$\theta_d$} (filter);

  \path[->,every node/.style={font=\sffamily\small}]
    (hubo-ach) edge node [left] {$\theta_r$} (filter);

  \path[->,every node/.style={font=\sffamily\small}]
    (filter) edge node [right] {$\theta_a$} (hubo-ach);


% look into this and add z^-1

\path [every node/.style={draw,minimum width=3cm, minimum height=5cm]}]
  node (a) at (0,0) {}
  [xshift=7cm]
  node (b) at (0,0) {}
  [xshift=7cm]
  node (c) at (0,0) {};

%\begin{scope}[->,>=latex]
%    \foreach \i in {-2,...,2}{% 
%      \draw[->] ([yshift=\i * 0.8 cm]a.east) -- ([yshift=\i * 0.8 cm]b.west) ;}

%    \foreach \i in {1,2}{% 
%      \draw[->] ([yshift=\i * 0.8 cm]a.east) to [out=50,in=130] ([yshift=\i * 0.8 cm]c.west) ;} 

%    \foreach \i in {-1,-2}{% 
%      \draw[->] ([yshift=\i * 0.8 cm]a.east) to [out=-50,in=-130] ([yshift=\i * 0.8 cm]c.west) ;}
%\end{scope}


\end{tikzpicture}
\caption{Desired reference $\theta_d$ being filtered before applied to Hubo via Hubo-Ach.  $\theta_d$ is sent through a filter that reduces the \textit{jerk} on the actuator then the new reference $\theta_r$ is set on the \textbf{FeedForward} channel, Hubo-Ach reads it then commands Hubo at the rising edge of the next cycle.}
\label{fig:hubo-ach-feedforward}
\end{figure}



Fig.~\ref{fig:singleJointStepFilteredFeedback} shows $\theta_r$ plotted against $\theta_c$ and $\theta_a$.  
$\theta_a$ not only lags behind $\theta_c$ but it also has a greater steady state error.
Fig.~\ref{fig:singleJointStepFilteredFeedbackMoment} shows how the steady state error $\theta_e^{fbfilter}$ increases with an applied moment.
This is where we get our compliance.

\begin{figure}[thpb]
  \centering
\includegraphics[width=0.8\columnwidth]{./pix/tmp.png}
  \caption{$\theta_r$ plotted against $\theta_c$ and $\theta_a$ recorded via Hubo-Ach using the feedback filtering method.}
  \label{fig:singleJointStepFilteredFeedback}
\end{figure}

\begin{figure}[thpb]
  \centering
\includegraphics[width=0.8\columnwidth]{./pix/tmp.png}
  \caption{$\theta_r$ plotted against $\theta_c$ and $\theta_a$ recorded via Hubo-Ach using the feedback filtering method with different moments applied to the joint.  You will note that as the moment increases so does $\theta_e^{fbfilter}$. }
  \label{fig:singleJointStepFilteredFeedbackMoment}
\end{figure}
