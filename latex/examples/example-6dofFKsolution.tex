The transform between joint adjacent joints is represented by the transform:

\begin{equation}\label{eq:dhT}
T_{i-1}^{i} = \left[ \begin{array}{cccc} 
cos(\theta_i) & -sin(\theta_i)cos(\alpha_i) &  sin(\theta_i)sin(\alpha_i)  &  a_i cos(\theta_i) \\ 
sin(\theta_i) &  cos(\theta_i)cos(\alpha_i) & -cos(\theta_i)sin(\alpha_i)  &  a_i sin(\theta_i) \\
0             &  sin(\alpha_i)              &  cos(\alpha_i)               &  d_i               \\
0             &  0                          &  0                           &  1                 
\end{array} \right]
\end{equation}

Where $\theta_i$, $\alpha_i$ and $d_i$ are shown in Fig.~\ref{fig:dhDiagram}.  The coordinate frame of the Hubo2+ used for both the forward and inverse kinematics are defined in Fig.~\ref{fig:IkFkCoordinate} and Fig.~\ref{fig:IkFkCoordinateArm}.  The DH parameters for the arm for the transform $T_i^{i-1}$ is found in Table~\ref{table:dhparamrightArm}.

\begin{figure}[thpb]
  \centering
%  \begin{minipage}{\textwidth}
\includegraphics[width=0.5\columnwidth]{./examples/pix/Sample_Denavit-Hartenberg_Diagram.png}
\caption{Denavit-Hartenberg diagram showing that axis of rotations and displacements to create the transform in Equation~\ref{eq:dhT}.  $\alpha$ is the angle between the axis of rotation of joint $n$ and $n-1$ about the of $n$. $\theta$ is the angle between the axis of rotation of joint $n$ and $n-1$ about the axis perpendicular to the axis about $n$.}
Image Credit: \textit{http://en.wikipedia.org/wiki/File:Sample\_Denavit-Hartenberg\_Diagram.png}
\label{fig:dhDiagram}
%  \end{minipage}
\end{figure}


\begin{figure}[thpb]
  \centering
%  \begin{minipage}{\textwidth}
\includegraphics[width=0.8\columnwidth]{./examples/pix/hubo2GTechFrame.pdf}
\caption{Hubo2+ coordinate frame for use with the forward and inverse kinematic example.  These coordinate frames are defined specifically for the IK and FK examples and are the same frame as in\cite{5649842}}
\label{fig:IkFkCoordinate}
%  \end{minipage}
\end{figure}

\begin{figure}[thpb]
  \centering
%  \begin{minipage}{\textwidth}
\includegraphics[width=0.8\columnwidth]{./examples/pix/parkIKHuboArm.png}
\caption{Hubo2+ coordinate frame for right arm.  Uses with the forward and inverse kinematic example.  These coordinate frames are defined specifically for the IK and FK examples and are the same frame as in\cite{5649842}   }
\label{fig:IkFkCoordinateArm}
%  \end{minipage}
\end{figure}



\begin{table}
\centering
\caption{Denavit–Hartenberg Parameters (continued) for Hubo2+ upper body (arms) in standard format}
\begin{tabular}{| c || c | c | c | c|}
\hline
$i$ (frame)  & $\theta_i$ (rad)         & $\alpha_i$ (rad) & $a_i$ (m) & $d_i$ (m) \\
\hline
\hline
1            & $\theta_1+\frac{\pi}{2}$ & $\frac{\pi}{2}$  & 0         & 0          \\
\hline
2            & $\theta_2-\frac{\pi}{2}$ & $\frac{\pi}{2}$  & 0         & 0          \\
\hline
3            & $\theta_3+\frac{\pi}{2}$ & $-\frac{\pi}{2}$ & 0         & $-l_{A2}$  \\
\hline
4            & $\theta_4$               & $\frac{\pi}{2}$  & 0         & 0          \\
\hline
5            & $\theta_5$               & $\frac{\pi}{2}$  & 0         & $-l_{A3}$  \\
\hline
6            & $\theta_6+\frac{\pi}{2}$ & 0                & $l_{A4}$  & 0          \\

\hline

\end{tabular}\label{table:dhparamrightArm}
\end{table}

In order to calculate for full FK transform $T_{N}^{E}$, where $E$ represents the end-effector and $N$ is the neck (robot origin), we must first calculate:

\begin{equation}\label{eq:t06}
T_0^6 = \prod_{i=1}^{6} T_{i-1}^{i} = T_{0}^{1}T_{1}^{2}T_{2}^{3}T_{3}^{4}T_{4}^{5}T_{5}^{6}
\end{equation}

In order to procure the transform $T_{N}^{E}$ we must pre-multiply $T_{0}^{6}$ by the transform $T_{N}^{0}$ and post-multiply it by the transform $T_{6}^{E}$.  
This results in transform $T_{N}^{E}$:

\begin{equation}
T_{N}^{E} = T_{N}^{0}T_{0}^{6}T_{6}^{E}
\end{equation}

Where $T_{N}^{0}$ is

\begin{equation}
T_{N}^{0} = \left[ \begin{array}{cccc} 
0 & 0 & 1 &  l_{A1} \\ 
1 & 0 & 0 &  0        \\
0 & 1 & 0 &  0        \\
0 & 0 & 0 &  1                 
\end{array} \right]
\end{equation} 


Now with a given set of joint space angles we can find the end-effectors position in reference to the robot origin, the neck.

