The degrees of freedom (DOF) of robots and complex systems have been increasing increasing exponentially since the early 20$^{th}$ century.
Today it is common place for complex control systems to have 40 DOF. 
This number is projected to be 70 DOF by the year 2020.
Robots with high DOF allows for complex tasks such as tool manipulation\cite{lofaroRAM2013,lofaroTePRA2013HuboAch,lofaroTePRA2013Valve,gtechIK}, greater human-robot interaction such as perform music\cite{lofaroEURASIP2011, 6094987,lofaroIASTED2011,5686847} and agile full-body locomotion\cite{lofaroHumanoids2012,lofaroGamesRobot,tepraLadder2013}.
More DOF require greater attention to:
\begin{itemize}
\item local communication delays
\item bandwidth
\item system configuration
\item stability.
\end{itemize}
In addition different tasks being performed by separate parts of the robot in tandem bring on greater issues including controller timing and priorities.
The increase in DOF on single system requires that the traditional methods of controller design be re-examined.

Due to the large entry cost of high DOF robots the controller should be able to be tested on no/low cost to entry systems.
This means the controller must be compatible with:
\begin{itemize}
\item virtual/simulated robot
\item kinematicly scaled robot
\item full-size robot
\end{itemize}

It is evident that the critical gap is needing a \textit{unified algorithmic framework for high degree of freedom robots that allows for development on multiple platforms}.
This unified algorithmic framework connects the three robots above in the \textit{Three Tier Infrastructure}\cite{threeTier} for complex system development as described in Section~\ref{sec:threeTier}.
The three tiers include:
\begin{itemize}
\item Rapid Prototype (RP) phase with zero cost to entry (OpenHubo Platform Section~\ref{sec:openhubo})
\item Test and Evaluation (T\&E) phase with low cost to entry (Mini-Hubo Platform Section~\ref{sec:mini-hubo})
\item Verify and Validate (V\&V) phase with lease-time cost to entry (Hubo Platform Section~\ref{sec:hubo})
\end{itemize}
The unifying algorithmic framework called \textit{Hubo-Ach}\cite{lofaroRAM2013} is described in Section~\ref{sec:hubo-ach}.


This work demonstrates that a multi-process, multi-rate control structure coupled with the proper timing mechanisms is conducive to creating this unified algorithmic framework.
Through verification and validation Hubo-Ach is shown to be a viable unifying algorithmic framework conducive to collaborative work.  

A road map of how this work began is shown in Section~\ref{sec:roadmap}.

An example of the three tier infrastructure being used to enable a high DOF robot to throw a ball is given in Section~\ref{sec:baseball}.
The methods used include a unique algorithm for end-effector velocity control called Sparse Reachable Maps (SRM)(Section~\ref{sec:sec:srm}).
One end-effector velocity control method is used in a live throwing experiment at a baseball game (Section~\ref{sec:finalDesign}).

The Hubo-Ach system is verified under many circumstances including:
\begin{itemize}
	\item Real-time closed form inverse kinematic controller (Section~\ref{sec:hubo-ach-kinimatics})
	\item Full body locomotive task of turning a valve (Section~\ref{sec:valve})
	\item Full body locomotive task of walking (Section~\ref{sec:OpenHuboWalking})
	\item \textit{Challenge: Aiming the throw:} Answer: Visual seroving while performing full body locomotive task (Section~\ref{sec:visuralServoing})
	\item \textit{Challenge: Safely landing when throwing:} Answer:Active damping via force-torque feedback (Section~\ref{sec:activedamping})
\end{itemize}




Hubo-Ach is then independently validated by other researcher through the examples of:
\begin{itemize}
	\item Door opening (Section~\ref{sec:hubo-achVerification})
	\item Dynamic walking (Section~\ref{sec:dynamicWalking})
\end{itemize}


A study/survey about how well the Hubo-Ach system performs as a \textit{unifying algorithmic framework} is given.  
Results and the questions are given in Section~\ref{sec:hubo-achSurvey}.

Lastly Section~\ref{sec:conclusion} discusses the results of the work and the future of this system.

\noindent \textit{Note:} This work has already been validated by pears in the field through:
\begin{itemize}
\item Use as the primary control system for the DARPA Robotics Challenge Track-A Team DRC-Hubo, Section~\ref{sec:drc}.
\item Used in the NSF-MIRR project\footnote{NSF-MIRR: Major Research Infrastructure Recovery and Reinvestment (MIRR) \#CNS-0960061 sponsored by the the U.S. National Science Foundation (NSF)}.
\item Various research being conducted using Hubo-Ach at MIT, WPI, Purdue, Ohio State, Swarthmore College, Georgia Tech, and Drexel University.
\end{itemize}
\noindent In addition more information on \textit{why this problem is hard} is in Section~\ref{sec:challenges}.

For the remainder of this document the focus will be on implementing this Unifying Algorithmic Framework on the different platforms of the three tier infrastructure.
These platforms are Hubo, Mini-Hubo, and OpenHubo.  
Detailed description of each of these robots are available in Section~\ref{sec:robots}.








































