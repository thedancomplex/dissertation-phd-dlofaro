The degrees of freedom (DOF) of robots and complex systems have been increasing increasing exponentially since the early 20$^{th}$ century.
Today it is common place for complex control systems to have 40 DOF. 
This number is projected to be 70 DOF by the year 2020 (see Section~\ref{sec:numdof}).
Robots with high DOF allows for complex tasks such as tool manipulation\cite{lofaroRAM2013,lofaroTePRA2013HuboAch,lofaroTePRA2013Valve,gtechIK}, greater human-robot interaction such as music performances\cite{lofaroEURASIP2011, 6094987,lofaroIASTED2011,5686847} and agile full-body locomotion\cite{lofaroHumanoids2012,lofaroGamesRobot,tepraLadder2013}.
More DOF require greater attention to:




\begin{multicols}{2}
\begin{itemize}
\item local communication delays
\item bandwidth
\item system configuration
\item stability
\end{itemize}
\end{multicols}


In addition different tasks being performed by separate parts of the robot in tandem bring on greater issues including controller timing and priorities.
The increase in DOF on a single system requires that the traditional methods of controller design be re-examined.

Experimental results in kinematic planning (Section~\ref{sec:KinematicPlaningBackground}), end-effector velocity control (Section~\ref{sec:baseball}) and human-robot interaction\cite{5686847} resulted in specific additional requirements for a high DOF controller for complex systems and humanoids. 
These requirements include:
\begin{multicols}{2}
\begin{itemize}
\item Robust controller integration 
\item High gain position controlled joints move without creating an \textit{over torque} condition
\item Live control
\item Synchronous control
\item Run on onboard computer
\item Run in real-time
\item Allow for hardware out of the loop testing
\item Refined programming methods
\end{itemize}
\end{multicols}

This work describes the creation of a controller architecture for high DOF robots that achieves all of the above requirements.
The system is call Hubo-Ach and has the following key attributes: 
%\begin{multicols}{2}
\begin{itemize}
\item Real-Time Performance
\item Inherently robust controller integration via multi-process architecture 
\item No-Head of Line Blocking scheme (newest data first)
\item Low CPU usage (\textit{lean and mean}, written in C)
\item Compatible with almost any simulator
\item C/C++, Python and Matlab bindings
\item Built-in real-time networking support (up to 1khz)
\item Maximum limiting bus bandwidth
\item Robot agnostic
\end{itemize}
%\end{multicols}

Hubo-Ach is verified via comprehensive experiments.
It is validated via third party implementation. 

The Hubo-Ach system is described in detail in Section~\ref{sec:hubo-ach}.
Full documentation on usage and programming examples of Hubo-Ach is given in Section~\ref{sec:huboAchManual}.
Verification of Hubo-Ach performance is given in Section~\ref{sec:experiment2}.
Third party validation of Hubo-Ach performance is given by Zucker et. al.\cite{tepraDoor2013}, O'Flasherty et. al.\cite{tepraCut2013}, and Section~\ref{sec:dynamicWalking}.
Finally a survey of 17 independent Hubo-Ach users showing overwhelming positive results is given in Section~\ref{sec:hubo-achSurvey}.


































