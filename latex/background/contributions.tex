My thesis came into being through a path created from my past research.
I started with the goal of having a humanoid robot become an interactive musical participant with humans.
I created a visual method of tracking the beat in the absence of auditory cues\cite{5686847}.
This came from a modification of a method of allowing children to play interactive games with humanoid robots\cite{lofaroGamesRobot}.
This method was effective, but to increase the accuracy I combined a pre-existing auditory beat tracker with my system.
This calumniated with a multi process system that combine the auditory and visual beat trackers\cite{lofaroIASTED2011,6094987,lofaroEURASIP2011}.
A human comparison was completed and found that this combined method was as accurate at detecting the beat in music as average humans.
From this I learned my \textbf{first lesson}:
\begin{adjustwidth}{2cm}{} \small
%\begin{danindent} \small 
\noindent \textit{When collaborating with other to create a complex robot control systems integrating controllers is difficult and causes many problems due to loop rates, library conflicts and stability.
I found that it is best to keep working systems independent allowing them to run at their native rate and on their native platforms.}
%\end{danindent} \normalsize
\end{adjustwidth} \normalsize
\noindent This was the start of the road towards my final thesis.
I then changed gears a little and moved to kinematic planning and end effector velocity control. 
I developed a method that is able to solve inverse kinematics (IK) for high degree of freedom (DOF) systems where there is no closed-form solution as well as create collision free trajectories for high DOF robots\cite{6385987}.
This is described in detail in Section~\ref{sec:srm}.
This culminated in the development of making the Hubo full-size humanoid robot throw the first pitch at a Major League Baseball (MLB) game\cite{lofaroHumanoids2012,6462956}.
From this I learned my \textit{second lesson}:
\begin{adjustwidth}{2cm}{} \small
\noindent \textit{When controllers and planners it is important that low-level controllers such as balance and obstacle avoidance run at all times. 
Non-priority controllers such as throwing trajectory planning can run in the background in a separate process.
Keeping the processies separate allowed the system to be more resistant to lag and crashes of one or more of the controllers. }
\end{adjustwidth} \normalsize
\noindent 
At this point I had \textit{hacked} together pre-existing systems that allowed the robot to do what I wanted it to do.  
I learned a few lessons along the way.
This is the point where I found that to make further impact in the field a \textit{Control Archetecture for High Degree of Freedom Complex Systems}, specifically humanoids, needs to be created.
From the lessons I have learned I knew that:
\begin{itemize}
\item Must inherently decouple controllers loop rates and phases
\item Must allow for collaborators not have to \textit{inject} their code into existing source.
\end{itemize}
\noindent This is where Hubo-Ach was born.
The idea was to create a multi process architecture for humanoid control using state of the art high-speed low-latency Inter-Process Communication (IPC) techniques\cite{lofaroRAM2013}.
The need for this system became even greater when the Hubo was chosen to be the primary platform for the DRC-Hubo\footnote{DRC-Hubo: http://www.drc-hubo.com/} Track-A team.
Since its initial conception Hubo-Ach has become a fully functional system used in active research by multiple universities including MIT, WPI, Purdue, Ohio State, Swarthmore College, Georgia Tech, and Drexel University\cite{lofaroTePRA2013HuboAch,lofaroTePRA2013Valve}.

\begin{itemize}
\item Developed a multi-process control system for humanoid robots using the Ach IPC.
\item Developed a method to solve inverse kinematics (IK) for high degree of freedom (DOF) systems.
\item Used IK methods on full-size humanoid robot making it throw the first pitch at a Major League Baseball (MLB) game for a live experiment
\end{itemize}