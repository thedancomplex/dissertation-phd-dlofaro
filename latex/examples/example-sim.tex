The simulator used for Hubo-Ach is the OpenHubo.
OpenHubo is an open-source kinematic and dynamic simulator for the the Hubo2 and Hubo2+ series robots.
It was developed by the Drexel Autonomous Systems Lab and runs using the open-source robot simulation environment OpenRAVE\cite{diankovThesis}.
Fig.~\ref{fig:openhubbo} shows the OpenHubo shell model and collision model.

\begin{figure}[thpb]
  \centering
      \includegraphics[width=0.4\columnwidth]{./pix/hBody.png}\includegraphics[width=0.4\columnwidth]{./pix/hCol.png}
      
\caption{OpenHubo model of the Hubo2 humanoid robot developed by the Drexel Autonomous Systems Lab and runs using the open-source robot simulation environment OpenRAVE\cite{diankovThesis}.  (Left) Shell Model - High polygon count.  (Right) Collision model - Made with primitives.}
\label{fig:openhubbo}
\end{figure}

The masses and lengths are of the OpenHubo model are all based off of the CAD model.
The shell model includes an external skin based off of the CAD model of the Hubo's shell.
This model is high polygon count and thus tends to require more processing time to detect collisions.
The collision model is constructed out of primitives in order to decrease the complexity of the model and decrease required processing time.
The collision model is a representation of the shell model. 
It does not exactly fit the contours but through experimentation and use has been calibrated to be a good representation of the Hubo's outer shell.


\begin{figure}
\centering

\begin{tikzpicture}[->,>=stealth',shorten >=1pt,auto,node distance=5cm,
  thick,main node/.style={fill=white!20,draw,font=\sffamily\Large\bfseries}]


  \node[main node] (ctrl) {Controller};
  \node[main node] (filter) [right=1.5cm of ctrl] {Filter};
  \node[main node] (hubo-ach) [below=1.0cm of filter] {Hubo-Ach};
  
  \node[main node,font=\small] (hold1) [right=1.5cm of hubo-ach, yshift=0.5cm] {hold};
  \node[main node,font=\small] (hold2) [right=1.5cm of hubo-ach, yshift=-0.5cm] {hold};

  \node[main node] (hubo) [right=1.5cm of hold1, yshift=-0.5cm] {OpenHubo};




%  \path[->, every node/.style={font=\sffamily\small}]
%    (hubo-ach) edge node [above] {$\theta_c$} (hubo);

\draw[->] ([yshift=0.2 cm]hubo-ach.east)  to [out=0,in=-180] node [below] {$\theta_c$} ([yshift=-0.0 cm]hold1.west)  ;
\draw[->] ([yshift=0.0 cm]hold1.east)  to [out=0,in=-180] node [below] {$\theta_c$} ([yshift=0.2 cm]hubo.west)  ;
\draw[-*] ([xshift=1.0 cm]hubo-ach.north)  to [out=60,in=120] node [above] {$\Gamma_{ts}$} ([yshift=-0.05 cm]hold1.north)  ;



\draw[->] ([yshift=0.0 cm]hold2.west)  to [out=180,in=0] node [below] {$H_{state}$} ([yshift=-0.2 cm]hubo-ach.east)  ;
\draw[->] ([yshift=-0.2 cm]hubo.west)  to [out=180,in=0] node [below right] {$H_{state}$} ([yshift=0.0 cm]hold2.east)  ;
\draw[-*] ([xshift=0.0 cm]hubo.south)  to [out=-120,in=-60] node [above] {$\Gamma_{fs}$} ([yshift=0.05 cm]hold2.south)  ;

\draw[->] ([yshift=-0.0 cm]hubo-ach.west)  to [out=180,in=-90] node [below left] {$H_{state}$} ([yshift=0.0 cm]ctrl.south)  ;



%\draw[->] ([yshift=-0.2 cm]hubo.west)  -- node [below] {$H_{state}$} ([yshift=-0.2 cm]hubo-ach.east)  ;
%\draw[->] ([yshift=-0.0 cm]hubo.south)  to [out=-120,in=-60] node [below] {$\Gamma_{fs}$} ([yshift=-0.0 cm]hubo-ach.south)  ;



  \path[->,every node/.style={font=\sffamily\small}]
    (ctrl) edge node [above] {$\theta_d$} (filter);

 \draw[->] ([xshift=-0.5 cm]filter.south)  -- node [left] {$\theta_r$} ([xshift=-0.5 cm]hubo-ach.north)  ;
 \draw[->] ([xshift=0.5 cm]hubo-ach.north) -- node [left] {$\theta_a$} ([xshift=0.5 cm]filter.south)  ;


\end{tikzpicture}
\caption{Diagram of how the OpenHubo simulator is connected to Hubo-Ach.  No changes to previous controllers are required for them to work with the simulator.  Just as before the desired reference $\theta_d$ being filtered before applied to Hubo via Hubo-Ach.  $\theta_d$ is sent through a filter that reduces the \textit{jerk} on the actuator then the new reference $\theta_r$ is set on the \textbf{FeedForward} channel, Hubo-Ach reads it then commands Hubo at the rising edge of the next cycle.  At this point $\Gamma_{ts}$ is set high and the OpenHubo simulator reads $\theta_c$.  The reference is set within OpenHubo and solved with a simulation period of $T_{sim}$.  Once The state, $H_{state}$ has been determined it is placed on the Hubo-Ach \textbf{FeedForward} channel and the ready trigger $\Gamma_{fs}$ is raised.  Hubo-Ach is waiting for the rising edge of $\Gamma_{fs}$ to continue on to the next cycle.}
\label{fig:openhubosim}
\end{figure}





Hubo-Ach is designed to run in either real-time or in simulation time.
This is especially useful when running experimental controllers and you do not want to damage the robot.
In real-time mode Hubo-Ach updates at its normal sampling period $T_{r}$.
Due to the complexity of the 
