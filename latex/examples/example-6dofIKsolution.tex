The next step is to find the inverse kinematic (IK) solution for the righ arm.
Inherently this problem has multiple solutions.
When solving the IK Pieper\cite{peiper1968kinematics} states that a closed-form solution does exist if:
\begin{itemize}
\item Three consecutive joints axes of the manipulator are parallel to one another
\end{itemize}

OR
\begin{itemize}
\item Three consecutive joints intercect at a single point
\end{itemize}

The kinematic structure in Fig~\ref{fig:hubo} and Fig~\ref{fig:IkFkCoordinate} shows that the Hubo2+ platform does have a three joints that intersect the same point in the shoulders and in the hips.
Thus a closed-form solution exists for both arms and both legs.

The transform $T_0^6$ in Equation~\ref{eq:t06} is needed to solve the IK problem for the shoulder.  
It is important to note that $T_0^6$ is in the form of

\begin{equation}
T_0^6 = \left[ \begin{array}{cccc} 
\overline{x_6} & \overline{y_6} & \overline{z_6} & \overline{p_6} \\
0              & 0              & 0              & 1   
\end{array} \right]
\end{equation} 

Where $\overline{x_6}$, $\overline{y_6}$ and $\overline{z_6}$ are $[3x1]$ unit vectors along the principle axes of the end-effectors frame, see Fig.~\ref{fig:IkFkCoordinate}.
Position vector $\overline{p_6}$ describes the hand about joint $A1$ (shoulder).
The arm can be vied in different frames.
If we look at the arm in reference to the end-effector's frame.
The reverse transform is defined as $(T_0^6)`$


\begin{equation}
(T_0^6)` = T_6_0 = (T_ = \left[ \begin{array}{cccc} 
\overline{x_6} & \overline{y_6} & \overline{z_6} & \overline{p_6} \\
0              & 0              & 0              & 1   
\end{array} \right]
\end{equation} 
