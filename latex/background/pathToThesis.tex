This section provides context to the origin of the idea of a unified algorithmic framework for complex systems called Hubo-Ach.

%%%%%%% Put what I wrote here

In summer 2008 Daniel participated in the NSF-EAPSI (East Asian and Pacific Summer Institute) allowing him to study at the Hubo-Lab at KAIST to learn how to maintain, operate and program the Hubo series robot.
In Fall of 2008 Daniel started his work on a Hubo KHR-4 model in the Drexel Autonomous Systems Lab (DASL) at Drexel University.
Shortly after that in Spring of 2009 he had the robot performing interactive musical tasks such as listening to music and autonomously tapping its hand to the beat\cite{5686847}.
This was the first of many experiments in sensor integration on the Hubo.
Later that spring Daniel and the rest of DASL showed Hubo to the public at a live demonstration at the Philadelphia Please Touch Museum. 
In winter of 2009 the first of the visual feedback methods was implemented \cite{lofaroGamesRobot}.
In 2010 Daniel investigated brain machine interfacing with the robot as well as multi-modal sensing using visual and auditory cues\cite{lofaroIASTED2011}.
In late 2010/early 2011 Daniel started on his first throwing experiments which culminated in making the Hubo robot throw the first pitch at a Major League Baseball game\cite{lofaroHumanoids2012}.  
A timeline of Daniel's work can be seen in Fig.~\ref{fig:timeline}.

\begin{figure}[thpb]
  \centering
\includegraphics[angle=90, width=0.9\columnwidth]{./pix/Timeline.pdf}
  \caption{Timeline of Daniel M. Lofaro's research from 2008 to 2012}
  \label{fig:timeline}
\end{figure}

Throughout this work Daniel quickly realized that there was no simple and robust way of integrating controllers on top of the existing Hubo control system.
Hubo's original control system was written by Hubo-Lab in the Windows environment utilizing the Real-Time Extension (RTX) for Windows API.
This controller is a typical single loop, single process, real-time controller that gets its high level input via flags and data fields located in shared memory.
As the controller gets more complex it became more and more probable that something would throw a fault.  
If one part of the controller failed the whole system fails.



The Daniel worked with the Drexel Autonomous Systems Lab to create a linux based controller for the Hubo called ACES/Conductor.
This controller designed by Sherbert et. al.\cite{aces} is a multi-threaded real-time controller that breaks each joint into individual devices. 
Each of these devices has multiple layers including the hardware, control, and command layer.
Each of these layers runs their own real-time loop, sharing data via pointer passing for dynamic memory fields.
Though the theoretical concept for this controller was sound, proper implementation would be suitable for an FPGA, GPU or other processors with many cores.
Because each device had multiple real-time loops associated with it (one for each layer) and there are many devices (joints) on the Hubo the CPU usage was high.
When running on a dual core 1.6 Ghz Intel Atom Processor a constant 100\% CPU usage was recorded.
This did not allow other processies to run in tandem.
In addition there was an inherent memory leak in the dynamic memory.
This was brought about from the system never being able to guarantee that another thread is not using a block of memory, thus it is never trashed.
This caused the memory usage to increase at a predictable rate.
This would cause a complete system crash when the memory size surpassed the available memory of the system.
It was found that this system was not suitable to run the Hubo robot.

Learning from the past experiences Daniel sought out to create a controller that could:
\begin{multicols}{2}
\begin{itemize}
\item handel high degrees of freedom
\item simple sensor integration
\item runs in real-time
\item low CPU usage
\end{itemize}
\end{multicols}

This goal was realized with the creation of Hubo-Ach as described in Section~\ref{sec:hubo-ach}. 
Further examples of why Hubo-Ach was created can be found in Appendix~\ref{sec:baseball}

\subsection{Human Robot Interaction Preliminary Experiments}
The initial goal was to have a humanoid robot become an interactive musical participant with humans.
This spawned the creation of a visual method of tracking the beat in the absence of auditory cues\cite{5686847}.
This came from a modification of a method of allowing children to play interactive games with humanoid robots\cite{lofaroGamesRobot}.
The resulting method was effective, but to increase the accuracy it was required to combine a pre-existing auditory beat tracker with the visual system.
This calumniated with a multi process system that combine the auditory and visual beat trackers\cite{lofaroIASTED2011,6094987,lofaroEURASIP2011}.
A human comparison was completed and found that this combined method was as accurate at detecting the beat in music as average humans.

\subsubsection{Results from preliminary experiments}
When collaborating with other to create a complex robot control systems integrating controllers is difficult because of the use of:
\begin{itemize}
\item different loop rates causing synchronization issues
\item different programming languages making using the same libraries a challenge
\end{itemize}

It was found that it is best to keep each working systems \textit{independent} allowing them to run at their native rate and on their native platforms\cite{ach}.



\subsection{High Degree of Freedom Kinematic Planning Preliminary Experiments}
The next challenge was to perform kinematic planning for end effector velocity control. 
This resulted in the development of a method that is able to solve inverse kinematics (IK) for high degree of freedom (DOF) systems where there is no closed-form solution as well as create collision free trajectories for high DOF robots\cite{6385987}.
This is described in detail in Section~\ref{sec:srm} and \ref{sec:baseball}.
This culminated in the verification and validation of the system by an experiment where Hubo full-size humanoid robot throw the first pitch at a Major League Baseball (MLB) game\cite{lofaroHumanoids2012,6462956}.

\subsubsection{Results from preliminary experiments}
As best practice when controllers and planners are implemented it is important that low-level controllers such as balance and obstacle avoidance run at all times\cite{lofaroRAM2013}. 
Non-priority controllers such as throwing trajectory planning can run in the background in a separate process.
Keeping the processes separate allowed the system to be more resistant to lag and crashes of one or more of the controllers.
This brought validation to the overarching plan for the unified algorithmic framework for complex systems and humanoid robots.




\subsection{Lessons Learned}

At this point creating these experiment it was required to \textit{hacked} together pre-existing systems that allowed the robot to do the task.
This is the point where it was realized that a \textit{unified algorithmic framework for complex systems and humanoid robots} was required for further development in the field.
Key lessons learned from these experiments were:
\begin{itemize}
\item Must inherently decouple controllers loop rates and phases
\item Must allow for collaborators not have to \textit{inject} their code into existing source.
\item Must work with multiple robots for testing, evaluation, validation, and verification.
\end{itemize}

\noindent This is where Hubo-Ach was born.
The idea was to create a multi process architecture for humanoid control using state of the art high-speed low-latency Inter-Process Communication (IPC) techniques\cite{lofaroRAM2013}.
This is different from traditional IPC techniques because of the lack of head of line (HOL) blocking and focus on low-latency.
Section~\ref{sec:ipc} gives further details and comparisons of different IPCs.


The need for this unified framework was amplified when the Hubo was chosen to be the primary platform for the DRC-Hubo\footnote{DRC-Hubo: http://www.drc-hubo.com/} Track-A team.
Since its initial conception Hubo-Ach has become a fully functional system used in active research by multiple universities including MIT, WPI, Purdue, Ohio State, Swarthmore College, Georgia Tech, and Drexel University\cite{lofaroTePRA2013HuboAch,lofaroTePRA2013Valve}.
This research also acts as a key source of verification and validation of the system.

