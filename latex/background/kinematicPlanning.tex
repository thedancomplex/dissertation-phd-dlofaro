%% Need more citations
% do parks paper
% do IK
Kinematic planning focuses on creating and testing valid trajectories for series kinematic manipulators.
The focus of this research is on high degree of freedom (DOF), high-gain, position controlled mechanisms.
High-gain position controlled mechanisms are the focus because the experimental platform used for this work is a that type of robot.
This limits the work because it is crucial that the joint-space acceleration profile is correct or the system will over-torque and shutdown.


The works are chosen as it pertains to end-effector velocity control.
Throwing and hitting are examples of end-effector velocity control.  
The goal is to have the end-effector moving at a specific rate in a specific direction.
In most cases it demands whole-body coordination to achieve a desired end-effector velocity.  
Whole-body coordination is different for planted robots and un-planted robots.  


\noindent \textit{Fixed robots} are robots where the base is attached to the ground or the base is significantly more massive then the manipulator.
Planted robots do not have to worry about balance consternates. 

\noindent \textit{Un-fixed robots} are robots that have an manipulator that is not significantly lighter then the base.  
In addition the robot is not physically attached to the ground.
This results in the robot needing to satisfy balance constraints.
In the static case if the robot satisfies the zero moment point (ZMP) criteria it will remain stable~\cite{5686276}.
When the manipulator moves quickly, as in the case of pitching or throwing, such upper-body motions if not coordinated with the lower-body, can cause the humanoid to lose balance.  



%The goal of this work is to show the creation of collision free trajectories for end-effector velocity control, the first step in our overarching goal of creating a system with the ability to throw objects and retain balance.  Towards this, Section~\ref{sec:selfCollision} will discuss our method of detecting self collisions.  Section~\ref{sec:rarea} describes the creation of the robot's sparse reachable map (SRM), a map in $R^3$ of the reachable points. through setting random values to the robot in joint space that takes into account joint limitations and self collisions.  Section~\ref{sec:trajGen} shows the creation of a throwing trajectory in $R^3$ and placing it withing the robot's reachable area using the SRM.  Section~\ref{sec:ik} explains the inverse kinematics used to convert the throwing trajectory in $R^3$ to joint space for this high degree of freedom humanoid robot.  Section~\ref{sec:trap} describes the creation of the approach from the initial pose to the starting pose of the throwing trajectory using a variant of trapezoidal motion control to keep within the actuators' physical limitations.  Section~\ref{sec:exp} features experiments to demonstrate the successful execution of this paper's goal.  Section~\ref{sec:conc} concludes the paper and comments on future work.
