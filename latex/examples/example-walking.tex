This section shows examples of how Hubo-Ach was used for stable walking.
Examples are given using:
\begin{itemize}
\item Hubo2+ (Physical Robot)
\item OpenHubo (Simulator)
\item RobotSim Hubo (Simulator)
\end{itemize}

section~\ref{sec:WalkingPatternGeneration} shows how the open-loop walking trajectory is created.

Section~\ref{sec:OpenHuboWalking} shows how the open loop walking trajectory is run in sim-time on OpenHubo using Hubo-Ach.
Section~\ref{sec:RobotSimWalking} shows how the open loop walking trajectory is run in sim-time on RobotSim using Hubo-Ach.
Section~\ref{sec:HuboWalking} shows the same walking trajectory running on the real Hubo hardware in real-time using Hubo-Ach.
It also shows the difference between running in sim-time and real-time.
Section~\ref{sec:dynamicWalking} shows the result of a five day \textit{hack-a-thon} using Hubo-Ach to add dynamic walking capability.





%% ---------------- Walking Pattern Generation ------------------------
\subsection{Walking Pattern Generation}\label{sec:WalkingPatternGeneration}
The walking pattern demonstrated in this section is generated based on the work of Park et. al.\cite{4115633}
A walking pattern is the way in which a legged robot, in this case two legged, moves its joints to create a walking gate while maintaining stability.
The walking pattern consists of two major phases:
\begin{itemize}
\item Single Support Phase (SSP)
\item Double Support Phase (DSP)
\end{itemize}

\noindent \textbf{Single support phase} is when one foot is on the ground.
This phase is when one leg moves from one stepping position to the other.
The ZMP must remain above the planted foot to guarantee stability.\\

\noindent \textbf{Double support phase} is when both feet are planted on the ground.  
When in this phase the ZMP moves from above one foot to the other along the stable area as seen in Fig.~\ref{fig:zmp}.




\begin{figure}[t]
  \centering
\includegraphics[width=0.5\columnwidth]{./examples/pix/huboZMPx.pdf}
  \caption{Hubo model diagram for ZMP walking in the $x$ direction (side view).  $b$ and $f$ are the step lengths for the left and the right foot.  $A$ defines the ankle.  $t_1$ is the time of the starting of the step, $t_2$ defines the landing of the stepping foot.  $P$ defines the hip location.  $\widetilde{x}$ defines the walking velocity.  The middle diagram depicts the SSP and the left and right diagrams show the DSP.}
  \label{fig:huboZMPx}
\end{figure}

Fig~\ref{fig:huboZMPx} and \ref{fig:huboZMPy} shows the walking pattern phases on a Hubo model in the $x$ and $y$ direction respectively.
In these figures $A_R$ and $A_L$ defines the left and right ankles respectively.  
$t_1$ is the time of the starting of the step, $t_2$ defines the landing of the stepping foot.  
$t_0$ defines time when the stepping foot is at peak step height.  
$P$ defines the hip location.  
$\widetilde{x}$ defines the walking velocity.
$\widetilde{y}$ defines the body sway velocity.  
The middle diagram depicts the SSP and the left and right diagrams show the DSP.



\begin{figure}[t]
  \centering
\includegraphics[width=0.7\columnwidth]{./examples/pix/huboZMPy.pdf}
  \caption{Hubo model diagram for ZMP walking in the $y$ direction (front view).  $A_R$ and $A_L$ defines the left and right ankles respectively.  $t_1$ is the time of the starting of the step, $t_2$ defines the landing of the stepping foot.  $t_0$ defines time when the stepping foot is at peak step height.  $P$ defines the hip location.  $\widetilde{y}$ defines the body sway velocity.  The middle diagram depicts the SSP and the left and right diagrams show the DSP.}
  \label{fig:huboZMPy}
\end{figure}


The walking patterns are generated creating a joint space trajectory with a period $T$ of $0.005~sec$.
The patterns keep the ZMP criteria described in Section~\ref{sec:zmp}.
These walking patterns are used to test the simulated robots and the physical robots.
Fig.~\ref{fig:huboZMPjointSpace} shows the joint space walking pattern verses time.

\begin{figure}[t]
  \centering
\includegraphics[width=0.5\columnwidth]{./pix/tmp.png}
  \caption{Joint space walking pattern.  The trajectory sampling period $T$ is $0.05~sec$.  Forward step length is $0.2~m$, sway velocity $\widetilde{y}$ is $0.062~\frac{m}{sec}$, and step period is $0.8~sec$.}
  \label{fig:huboZMPjointSpace}
\end{figure}





%% ---------------- Walking Pattern Generation ------------------------
\subsection{Walking Using OpenHubo Simulator and Hubo-Ach}\label{sec:OpenHuboWalking}
The walking pattern that was generated in Section~\ref{sec:WalkingPatternGeneration} was then applied to the OpenHubo system described in Section~\ref{sec:simulator} via the Hubo-Ach controller.
The walking pattern was applied in sim-time with a period $T_{sim}$ of $0.005~sec$.
The block diagram of the system using OpenHubo in sim-time for a walking trajectory is shown in Fig.~\ref{fig:openhubosimWalking}.

\begin{figure}
\centering

\begin{tikzpicture}[->,>=stealth',shorten >=1pt,auto,node distance=5cm,
  thick,main node/.style={fill=white!20,draw,font=\sffamily\Large\bfseries}]


  \node[main node] (ctrl) [text width=3cm] {Walking Pattern};
 
  \node[main node] (hubo-ach) [right=1.5cm of ctrl] {Hubo-Ach};
  
  \node[main node,font=\small] (hold1) [right=1.5cm of hubo-ach, yshift=0.5cm] {hold};
  \node[main node,font=\small] (hold2) [right=1.5cm of hubo-ach, yshift=-0.5cm] {hold};
  \node[main node,font=\small] (hold3) [below=0cm of ctrl, yshift=0.0cm] {hold};


  \node[main node] (hubo) [right=1.5cm of hold1, yshift=-0.5cm] {OpenHubo};




%  \path[->, every node/.style={font=\sffamily\small}]
%    (hubo-ach) edge node [above] {$\theta_c$} (hubo);

\draw[->] ([yshift=0.2 cm]hubo-ach.east)  to [out=0,in=-180] node [below] {$\theta_c$} ([yshift=-0.0 cm]hold1.west)  ;
\draw[->] ([yshift=0.0 cm]hold1.east)  to [out=0,in=-180] node [below] {$\theta_c$} ([yshift=0.2 cm]hubo.west)  ;
\draw[-*] ([xshift=1.0 cm]hubo-ach.north)  to [out=60,in=120] node [above] {$\Gamma_{ts}$} ([yshift=-0.05 cm]hold1.north)  ;
%\draw[-*] ([xshift=-0.02 cm]hubo-ach.north)  to [out=120,in=60] node [above] {$\Gamma_{ts}$} ([yshift=-0.05 cm]hold3.north)  ;



\draw[->] ([yshift=0.0 cm]hold2.west)  to [out=180,in=0] node [below] {$H_{state}$} ([yshift=-0.2 cm]hubo-ach.east)  ;
\draw[->] ([yshift=-0.2 cm]hubo.west)  to [out=180,in=0] node [below right] {$H_{state}$} ([yshift=0.0 cm]hold2.east)  ;
\draw[-*] ([xshift=-0.02 cm]hubo.south)  to [out=-120,in=-60] node [above] {$\Gamma_{fs}$} ([yshift=0.05 cm]hold2.south)  ;
\draw[-*] ([xshift=0.02 cm]hubo.south)  to [out=-115,in=-55] node [above] {$\Gamma_{fs}$} ([yshift=0.05 cm]hold3.south)  ;

\draw[-*] (hold3.north)  to [out=90,in=-90] node [above] {}(ctrl.south)  ;

%\draw[->] ([yshift=-0.0 cm]hubo-ach.west)  to [out=180,in=-90] node [below left] {$H_{state}$} ([yshift=0.0 cm]ctrl.south)  ;



%\draw[->] ([yshift=-0.2 cm]hubo.west)  -- node [below] {$H_{state}$} ([yshift=-0.2 cm]hubo-ach.east)  ;
%\draw[->] ([yshift=-0.0 cm]hubo.south)  to [out=-120,in=-60] node [below] {$\Gamma_{fs}$} ([yshift=-0.0 cm]hubo-ach.south)  ;



  \path[->,every node/.style={font=\sffamily\small}]
    (ctrl) edge node [above] {$\theta_r$} (hubo-ach);



\end{tikzpicture}
\caption{Diagram of how the OpenHubo simulator is connected to Hubo-Ach and is used to run a walking trajectory.  
The walking pattern generator ensures proper constraints on the velocity, acceleration and jerk and thus the filter seen in Fig.~\ref{fig:openhubosim} is not desired.  
$\theta_r$ is set directly on the \textbf{FeedForward} channel thus each joint will have the response as seen in Fig.~\ref{fig:singleJointStep} for each commanded reference command at each time step.
Hubo-Ach reads the \textbf{FeedForward} channel and commands Hubo at the rising edge of the next cycle.  
At this point $\Gamma_{ts}$ is set high and the OpenHubo simulator reads $\theta_c$.  
The reference is set within OpenHubo and solved with a simulation period of $T_{sim}$.  
Once The state, $H_{state}$ has been determined it is placed on the Hubo-Ach \textbf{FeedForward} channel and the ready trigger $\Gamma_{fs}$ is raised.  
Hubo-Ach is waiting for the rising edge of $\Gamma_{fs}$ to continue on to the next cycle.  
In order to keep with the sim-time the \textit{Walking Pattern} also waits for the rising edge of $\Gamma_{fs}$ to put the next desired reference on the \textbf{FeedForward} channel. }
\label{fig:openhubosimWalking}
\end{figure}



In Fig.~\ref{fig:openhubosimWalking} the OpenHubo simulator is connected to Hubo-Ach and is used to run the walking trajectory.  
The walking pattern generator ensures proper constraints on the velocity, acceleration and jerk and thus the filter seen in Fig.~\ref{fig:openhubosim} is not desired.  
$\theta_r$ is set directly on the \textbf{FeedForward} channel thus each joint will have the response as seen in Fig.~\ref{fig:singleJointStep} for each commanded reference command at each time step.
Hubo-Ach reads the \textbf{FeedForward} channel and commands Hubo at the rising edge of the next cycle.  
At this point $\Gamma_{ts}$ is set high and the OpenHubo simulator reads $\theta_c$.  
The reference is set within OpenHubo and solved with a simulation period of $T_{sim}$.  
Once The state, $H_{state}$ has been determined it is placed on the Hubo-Ach \textbf{FeedForward} channel and the ready trigger $\Gamma_{fs}$ is raised.  
Hubo-Ach is waiting for the rising edge of $\Gamma_{fs}$ to continue on to the next cycle.  
In order to keep with the sim-time the \textit{Walking Pattern} also waits for the rising edge of $\Gamma_{fs}$ to put the next desired reference on the \textbf{FeedForward} channel.




