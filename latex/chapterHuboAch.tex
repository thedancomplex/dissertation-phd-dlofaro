%\chapter{Control System for Complete Complex Autonomous Systems: Hubo-Ach}\label{sec:hubo-ach}
%\chapter{Thesis: Hubo-Ach - A Control System for Complete Complex Autonomous Systems}\label{sec:hubo-ach}
%\chapter{Hubo-Ach: Control System for Complete Complex Autonomous Systems}\label{sec:hubo-ach}
\chapter{Hubo-Ach: A Unified Algorithmic Framework for High DOF Robots}\label{sec:hubo-ach}


This section describes in detain the \textit{unified algorithmic framework for high degree of freedom complex systems and humanoid robots}.
An overview of the system is given in Section~\ref{sec:ach:overview}.
Timing and system testing is given in Section~\ref{sec:timing}
Validation examples are given in Section~\ref{sec:simpleExamples}, \ref{sec:hubo-ach-kinimatics} and \ref{sec:valve}.
Verification of Hubo-Ach from independent parties is given in Section~\ref{sec:hubo-achVerification} and results from surveys about the system is given in Section~\ref{sec:hubo-achSurvey}.


%% Hubo-Ach Inspiration
\section{Overview}\label{sec:ach:overview}

%% Hubo-Ach explanation
%% workign wth ros and other software that exists, why in C


\input{hubo-ach/hubo-ach-intro.tex}
\section{Inter Process Comunication Comparision}\label{sec:ipc}
	POSIX provides three main types of IPC: streams, datagrams and shared memory.  
A review of each is made before making a choice for desired message passing skeam.

\noindent \textbf{Streams:}\\
The IPC type \textit{stream} includes pipes, FIFOs, stream sockets, and TCP sockets.
All stream basted methods suffer from head of line (HOL) blocking which means older data \textbf{must} be read before newer data.
%In addition all streaming methods are exposed to file abstraction (read/write byte sequence).
For robotic applications we must be able to access the newest data imediately and read older data if needed.
This is a different paradime then typical streaming application because robots are real-time sensitive meaning the newest information holds more value to the overall system than the older data.

\noindent \textbf{Datagrams:}\\
POSIX \textit{datagrams} come in two major flavors, \textit{datagram sockets} and \textit{POSIX message queues}.
Datagram sockets are less likely to block the sender then streams.
The most important reason why datagrams are \textbf{not} a good solution for my application is that newer messages are lost if the buffer is filled.
Newer data is more important than older data in my control system thus this is not a viable option.

POSIX message queues are simular to datagrams sockets with the addition of message priorities.
Unlike datagram sockets if the buffer fills the POSIX message queues will block.
This will cause the application to stop processing until it is able to read/flush the old messages.
Thus simular to other methods mentioned this also suffers from HOL.

\noindent \textbf{Shared Memory:}\\
POSIX shared memory is very fast and allows access to the latest data by simply writing over a variable.
Though I have been advicating that the newest information is the most important, old information can not be discarded.
If using POSIX shared memory there is no way of recovering older data that might have been missed by a controller.

What is needed is a method of sharing data that is \textit{non-blocking} and as \textit{low-latancy} like shared memory, but still holds older data and uses an asyncronous IO scheme.
The asyncronous IO scheme is required so the controller is not locked to a set rate by the data transactionn method.
N. Dantam et. al.\cite{ach} shows that Asynchronous IO (AIO) might be approperiate for this application however the implimentaiton under Linux is not as mature as I require.
In addition N. Dantam shows that other IPC mechanism using select/poll/epoll/kqueue are widely used network server and help midigate but not totally removed the issue of HOL.
The primary problem being that that thought the sender will not block the reader must stil read the oldest data first.
The question now is what IPC mechanism will be suitable for my control system.

Upon investigation three major mechanisms are avaliable; Robot Operating System (ROS)\cite{ros}, Message Passing Interface (MPI)\cite{Gropp:1999:UMP:330577} and Ach\cite{ach}.
Though ROS is ubicuitious in the robotics and automation field the inherent latency and the non real-time (RT) guarantee due to the use of TCP/IP it is not a good choice.
MPI is ubiquitous in the high-proformance computing field.  It has full non-blocking capiabilities and is geared towards maximizing message throughput for networked clusters\cite{ach}.
Table~\ref{table:ipc} shows a comparision of a wide range of IPC methods.
A focus on reducing latency is not given.
Ach is a newer 

\begin{table}
\caption{Robot control system comparison}
\begin{tabular}{l || c | c | c | c | c | c}
\hline
System                          & Open        & POSIX           & Non          & Real      & Low         & Light \\
                                &      Source &       Complaint &     Blocking &      Time &     Latency & Weight\\
\hline
\hline
ROS                             & yes         & yes             & no           & no        & no          & no \\
\hline
Orocos Real-Time                & yes         & yes             & no           & yes       & yes         & no \\
                 Toolkit        &             &                 &              &           &             &    \\
\hline

Robotics Technology             & yes         & yes             & no           & yes       & yes         & no \\
                    Middleware  &             &                 &              &           &             &    \\
\hline
Microsoft Robotics              & no          & no              & no           & yes       & yes         & no \\
                   Studio       &             &                 &              &           &             &    \\
\hline
Aware2.0                        & no          & yes             & no           & yes       & yes         & yes \\
\hline
Hubo-Ach                        & yes         & yes             & yes          & yes       & yes         & yes \\



\hline
\end{tabular}
\label{table:robotOS}
\end{table}


\begin{table}
\caption{Inter Process Comunication Method Comparison}
\small
\begin{tabular}{l || c | c | c | c | c | c | c}
\hline
Inter-                          & Open        & POSIX           & Non          & Multiple    & Low         & Light   & Access \\
Process                         &      Source &       Complaint &     Blocking & Senders     &     Latency & Weight  & Old   \\
Comunication                    &             &                 &              & and         &             &         & Data          \\
Method                          &             &                 &              & Receivers   &             &         &           \\
\hline
\hline
Streams                         & yes         & yes             & no           & yes       & no          & yes     & yes \\
\hline
Datagram                        & yes         & yes             & no           & yes       & no          & yes     & yes \\
Sockets                         &             &                 &              &           &             &         &     \\
\hline

POSIX                           & yes         & yes             & no           & yes       & no          & yes     & yes\\
Message                         &             &                 &              &           &             &         &     \\
Queues                          &             &                 &              &           &             &         &     \\
\hline
Shared                          & yes         & yes             & yes          & yes       & yes         & yes     & no \\
Memory                          &             &                 &              &           &             &         &     \\

\hline
AIO                             & yes         & yes             & yes          & yes       & yes         & yes     & yes \\
\hline
CORBA                           & yes         & yes             & yes          & no        & yes         & yes     & yes \\
\hline
ROS                             & yes         & yes             & no           & yes       & no          & no      & no \\
\hline
Data                            & yes         & yes             & yes          & yes       & yes         & yes     & yes\\
Distribution                    &             &                 &              &           &             &         &     \\
Service                         &             &                 &              &           &             &         &     \\
\hline
Ach                             & yes         & yes             & yes          & yes       & yes         & yes     & yes\\


\hline
\end{tabular}
\label{table:ipc}
\normalsize
\end{table}


\section{Timing}\label{sec:timing}
	\begin{figure}
\centering

\begin{tikzpicture}[->,>=stealth',shorten >=1pt,auto,node distance=5cm,
  thick,main node/.style={fill=white!20,draw,font=\sffamily\footnotesize\bfseries}]

  
  \node[main node] (latesRef) [ text width=2.0cm, minimum height=1.5cm, align=center]   {Reference\\(Latest)};
  \node[main node] (ctrl)     [left=0.8cm of latesRef,minimum height=2.5cm, align=center]                           {Controller};
  \node[main node] (getRef)   [right=1.5cm of latesRef, text width=3.0cm, minimum height=1.1cm, align=center]      {Get Reference};
 

 % \node[main node] (fromsim)     [above=1.0cm of getRef, text width=3.0cm, minimum height=1.1cm, align=center]    {From Simulator\\ Hold (sim mode)};
 % \node[main node] (sim)         [above=1.0cm of fromsim, text width=3.0cm, minimum height=1.1cm, align=center]    {Simulator \\Trigger\\ (sim mode)};

 \node[main node, draw=white] (haname2)   [above=0.2cm of getRef, align=center]   {};

  \node[main node] (dodebug)     [below=1.0cm of getRef, text width=3.0cm, minimum height=1.1cm, align=center]    {User Commands\\ (debug)};
  \node[main node] (tosim)       [below=1.0cm of dodebug, text width=3.0cm, minimum height=1.1cm, align=center]    {To Simulator\\ Trigger};
  \node[main node] (filter)      [below=1.0cm of tosim, text width=3.0cm, minimum height=1.1cm, align=center]     {Filter};
  \node[main node] (setRef)      [below=1.0cm of filter, text width=3.0cm, minimum height=1.1cm, align=center]    {Set Reference};
  \node[main node] (reqSens)     [below=1.0cm of setRef, text width=3.0cm, minimum height=1.1cm, align=center]    {Request Sensors};
  \node[main node] (getSens)     [below=1.0cm of reqSens, text width=3.0cm, minimum height=1.1cm, align=center]   {Get Sensors};
  \node[main node] (canWait)     [below=1.0cm of getSens, text width=3.0cm, minimum height=1.1cm, align=center]   {Wait on CAN};
  \node[main node] (putState)    [below=1.0cm of canWait, text width=3.0cm, minimum height=1.1cm, align=center]   {Put State};
  \node[main node] (rthold)      [left=0.5cm of getSens, minimum height=1.1cm, align=center]   {Real-Time\\Hold};



  \node[main node] (latestState) [left=1.5cm of putState, text width=2.0cm, minimum height=1.1cm, align=center, yshift=-1.0cm]   {State\\(Latest)};
  \node[main node, draw=white] (haname)   [right=3.0cm of putState, align=center]   {};

  \node[main node] (robot)    [right=4.8cm of getSens, minimum height=5.5cm, align=center]    {Robot};
  

%H_ref.ref[jnt] = deg;
%H_ref.mode[jnt] = HUBO_REF_MODE_ENC_FILTER;

  \path[->, dashed,every node/.style={font=\sffamily\small}]
    (ctrl) edge node [above] {} (latesRef);

%  \path[->, every node/.style={font=\sffamily\small}]
%    (fromsim) edge node [right] {$t_0=0.011~ms$} (getRef);
%\draw[->] ([xshift=-1.2 cm]fromsim.south)  to [out=-90,in=90] node [right] {$t_0=0.011~ms$} ([xshift=-1.2 cm]getRef.north)  ;
%\draw[->] ([xshift=-1.2 cm]getRef.south)   to [out=-90,in=90] node [right] {$t_1=0.011~ms$} ([xshift=-1.2 cm]tosim.north)  ;
%\draw[->] ([xshift=-1.2 cm]tosim.south)   to [out=-90,in=90] node [right] {$t_2=0.011~ms$} ([xshift=-1.2 cm]filter.north)  ;
%\draw[->] ([xshift=-1.2 cm]filter.south)   to [out=-90,in=90] node [right] {$t_3=???~ms$} ([xshift=-1.2 cm]setRef.north)  ;
%\draw[->] ([xshift=-1.2 cm]setRef.south)   to [out=-90,in=90] node [right] {$t_4=???~ms$} ([xshift=-1.2 cm]reqSens.north)  ;
%\draw[->] ([xshift=-1.2 cm]reqSens.south)   to [out=-90,in=90] node [right] {$t_5=???~ms$} ([xshift=-1.2 cm]getSens.north)  ;
%\draw[->] ([xshift=-1.2 cm]getSens.south)   to [out=-90,in=90] node [right] {$t_6=0.011~ms$} ([xshift=-1.2 cm]putState.north)  ;
%\draw[->] ([xshift=0.0 cm]putState.west)   to [out=180,in=-90] node [below, yshift=-0.5cm, xshift=-0.2cm] {$t_7=0.011~ms$} ([xshift=0.0 cm]rthold.south)  ;


\draw[->] ([xshift=0.0cm]latesRef.east)  to [out=0,in=-180]  node [right]  {} ([xshift=0.0cm]getRef.west)  ;
%\draw[->] ([xshift=0.0cm]fromsim.south)  to [out=-90,in=90]  node [right]  {} ([xshift=0.0cm]getRef.north)  ;
\draw[->] ([xshift=0.0cm]getRef.south)   to [out=-90,in=90]  node [right]  {} ([xshift=0.0cm]dodebug.north)  ;
\draw[->] ([xshift=0.0cm]dodebug.south)  to [out=-90,in=90]  node [right]  {} ([xshift=0.0cm]tosim.north)  ;
\draw[->] ([xshift=0.0cm]tosim.south)    to [out=-90,in=90]  node [right]  {} ([xshift=0.0cm]filter.north)  ;
\draw[->] ([xshift=0.0cm]filter.south)   to [out=-90,in=90]  node [right]  {} ([xshift=0.0cm]setRef.north)  ;
\draw[->] ([xshift=0.0cm]setRef.south)   to [out=-90,in=90]  node [right]  {} ([xshift=0.0cm]reqSens.north)  ;
\draw[->] ([xshift=0.0cm]reqSens.south)  to [out=-90,in=90]  node [right]  {} ([xshift=0.0cm]getSens.north)  ;
\draw[->] ([xshift=0.0cm]getSens.south)  to [out=-90,in=90]  node [right]  {} ([xshift=0.0cm]canWait.north)  ;
\draw[->] ([xshift=0.0cm]canWait.south)  to [out=-90,in=90]  node [right]  {} ([xshift=0.0cm]putState.north)  ;
\draw[->] ([xshift=0.0 cm]putState.west) to [out=180,in=-90] node [below, yshift=-0.5cm, xshift=-0.2cm] {} ([xshift=0.0 cm]rthold.south)  ;
\draw[->] ([xshift=0.0 cm]rthold.north)  to [out=90,in=-120] node [left, yshift=-1.0cm, xshift=-0.8cm, rotate=90] {} ([xshift=0.0 cm]getRef.west)  ;
\draw[->] ([xshift=0.0cm]putState.south) to [out=-90,in=0]   node [right]  {} ([xshift=0.0cm]latestState.east)  ;




%\draw[->] ([yshift=0.75cm]fromsim.east)  to [out=-30,in=30] node [right] {$t_0=0.014~ms$} ([yshift=-0.75cm]fromsim.east)  ;
\draw[->] ([yshift=0.75cm]getRef.east)  to [out=-30,in=30] node [right] {$t_0=0.010~ms$} ([yshift=-0.75cm]getRef.east)  ;
\draw[->] ([yshift=0.75cm]dodebug.east)  to [out=-30,in=30] node [right] {$t_1=0.011~ms$} ([yshift=-0.75cm]dodebug.east)  ;
\draw[->] ([yshift=0.75cm]tosim.east)  to [out=-30,in=30] node [right] {$t_2=0.014~ms$} ([yshift=-0.75cm]tosim.east)  ;
\draw[->] ([yshift=0.75cm]filter.east)  to [out=-30,in=30] node [right] {$t_3=0.008~ms$} ([yshift=-0.75cm]filter.east)  ;
\draw[->] ([yshift=0.75cm]setRef.east)  to [out=-30,in=30] node [right] {$t_4=0.152~ms$} ([yshift=-0.75cm]setRef.east)  ;
\draw[->] ([yshift=0.75cm]reqSens.east)  to [out=-47,in=47] node [right] {$t_5=1.365~ms$} ([yshift=-0.75cm]getSens.east)  ;
\draw[->] ([yshift=0.75cm]canWait.east)  to [out=-30,in=30] node [right,align=center] {$t_6=3.000~ms$\\(hard timeout)} ([yshift=-0.75cm]canWait.east)  ;
%\draw[->] ([yshift=0.75cm]reqSens.east)  to [out=-30,in=30] node [right] {$t_5=???~ms$} ([yshift=-0.75cm]reqSens.east)  ;
%\draw[->] ([yshift=0.75cm]getSens.east)  to [out=-30,in=30] node [right] {$t_6=???~ms$} ([yshift=-0.75cm]getSens.east)  ;
\draw[->] ([yshift=0.75cm]putState.east)  to [out=-30,in=30] node [right] {$t_7=0.092~ms$} ([yshift=-0.75cm]putState.east)  ;
\draw[->] ([yshift=-0.75cm]rthold.west)  to [out=120,in=-120] node [above, yshift=1.5cm, xshift=0.3cm, rotate=90] {$\displaystyle t_{hold}=T-\sum_{i=0}^{8} t_i=0.348~ms$} ([yshift=0.75cm]rthold.west)  ;

\draw[->, dashed] ([xshift=0.0cm]latestState.west)   to [out=120,in=-90] node [right] {} ([xshift=0.0cm]ctrl.south)  ;
%\draw[->, dashed] ([xshift=0.0cm]sim.south)   to [out=-90,in=90] node [right] {} ([xshift=0.0cm]fromsim.north)  ;

\node[fit=(getRef)(putState)(rthold)(haname)(latestState)(haname2), draw,label={south:Hubo-Ach}] (hubo-ach) [minimum width=4.5cm] {};




\draw[->,loosely dotted] ([xshift=0.75cm]reqSens.south)  to [out=-30,in=-180] node [above] {} ([yshift=0.75cm]robot.west)  ;
\draw[<-,loosely dotted] ([xshift=0.0cm]getSens.east)  to [out=0,in=-180] node [above, xshift=1.7cm] {CAN} ([yshift=0.0cm]robot.west)  ;
\draw[<-,loosely dotted] ([xshift=0.75cm]canWait.north)  to [out=30,in=-180] node [above] {} ([yshift=-0.75cm]robot.west)  ;




%  \path[->,every node/.style={font=\sffamily\small}]
%    (ik) edge node [above] {$\overline{\theta_d}$} (filter);

% \draw[->] ([xshift=-0.5 cm]filter.south)  -- node [left] {$\overline{\theta_r}$} ([xshift=-0.5 cm]hubo-ach.north)  ;
% \draw[->] ([xshift=0.5 cm]hubo-ach.north) -- node [left] {$\overline{\theta_a}$} ([xshift=0.5 cm]filter.south)  ;

% \path[<->,dashed, every node/.style={font=\sffamily\small}]
%    (hubo) edge node [above] {CAN} (hubo-ach);


\end{tikzpicture}
\caption{Timing diagram of Hubo-Ach.  All times $t_*$ denote measured times each block takes to complete.  Tests were done on a 1.6Ghz Atom D525 Dual Core with 1GB DDR3 800Mhz memory running Ubuntu 12.04 LTS linux kernel 3.2.0-29 on a Hubo2+ utilizing a CAN bus running at 1Mbps baud.  Average CPU usage is 7.6\% using a total of 4Mb or memory.}
\label{fig:hubo-ach-timing}
\end{figure}



\section{CPU Usage}
	The CPU usage was analized while the Hubo-Ach controller was being used in the following states:
\begin{multicols}{2}
\begin{itemize}
\item Idle
\item Under open-loop control
\item Reading the sensors
\item Under closed-loop control
\end{itemize}
\end{multicols}

Fig.~\ref{fig:timing-getTrigger} shows the result of this test.
The results confirm that the CPU utilization stays within 0.3\% when idle and under closed loop control.
This means that the CPU utilization of Hubo-Ach is independent of the external control method.
Thus it will not add more to the CPU load under complex control schemes then under simple ones.
This makes it easy to model Hubo-Ach in when adding it to a CPU usage budget.


\begin{figure}[thpb]
  \centering
\includegraphics[width=0.6\columnwidth]{./timingData/cpu.pdf}
  \caption{CPU utilization for the Hubo-Ach process when 1) idle, 2) under open-loop control, 3) reading the sensors, and 4) under closed-loop control.  
It is important to note that the cpu utilization stays within 0.3\% when idle and under closed loop control.
This means that the CPU utilization of Hubo-Ach is independent of the external control method.
Thus it will not add more to the CPU load under complex control schemes then under simple ones.}
  \label{fig:timing-getTrigger}
\end{figure}

\section{Verification Experiments}\label{sec:simpleExamples}
This section contains step by step verification examples showing the controller for high DOF complex system functions properly with the hubo system.
All controllers are implemented using the multiple processes approach and includes all latencies found in Section~\ref{sec:timing}.

	\subsection{Joint Space Step Response}\label{sec:singlejointStep}
		This section shows the experimental and expected results of controlling a single joint via the Hubo-Ach system.
In this example the right shoulder pitch (RSP) is given a step input from 0.0 $rad$ to 0.4 $rad$.
The reference position $\theta_r$ is begin recorded as well as the actuator setpoint $\theta_c$ and the actual position of the joint $\theta_a$.
These definitions are also available in Table~\ref{table:recorded}

\begin{table}
\centering
\caption{States being recorded for the single joint step response test}
\begin{tabular}{l || c | c | c | c}
Signal      & Symbol     & Definition                    & Source      & Units \\
\hline
\hline
FeedForward & $\theta_r$ & Desired reference on the      & Hubo-Ach   & $rad$ \\
            &            & Hubo-Ach FeedForward Channel  &            &       \\
\hline
FeedForward & $\theta_c$ & Reference set to the actuator & Hubo-Ach   & $rad$ \\
\hline
Feedback    & $\theta_a$ & Actual position of joint as   & JMC        & $rad$ \\
            &            & measured from the encoders    &            &       \\
\hline
\end{tabular}
\label{table:recorded}
\end{table}



\begin{figure}[thpb]
  \centering
\includegraphics[width=0.8\columnwidth]{./pix/tmp.png}
  \caption{The commanded reference plotted against the actual reference recorded via Hubo-Ach and ground truth via CAN analyzing utilities.  In this plot the commanded reference is not automatically filtered by Hubo-Ach.  The commanded joint is the right shoulder pitch.}
  \label{fig:singleJointStep}
\end{figure}

Fig.~\ref{fig:singleJointStep} shows the results when a step input is applied and Hubo-Ach is in \textit{HUBO\_REF\_MODE\_REF\_FILTER} also know as pass-through mode.
This sets the what the desired reference on the \textbf{FeedForward} Hubo-Ach channel to the actuator's reference, i.e.:

\begin{equation}\label{eq:refrefmode}
 \theta_c(N) = \theta_r(N)
\end{equation}

Fig.~\ref{fig:hubo-ach-feedforward} shows the block diagram of the control setup.

\begin{figure}
\centering
\begin{tikzpicture}[->,>=stealth',shorten >=1pt,auto,node distance=5cm,
  thick,main node/.style={fill=white!20,draw,font=\sffamily\Large\bfseries}]


  \node[main node] (ref) {Reference};
  \node[main node] (hubo-ach) [right of=ref] {Hubo-Ach};
  \node[main node] (hubo) [right of=hubo-ach] {Hubo};




  \path[<->,dashed, every node/.style={font=\sffamily\small}]
    (hubo) edge node [above] {CAN} (hubo-ach);

  \path[->,every node/.style={font=\sffamily\small}]
    (ref) edge node [above] {$\theta_r$} (hubo-ach);


\end{tikzpicture}
\caption{Reference $\theta_r$ being applied to Hubo via Hubo-Ach.  $\theta_r$ is set on the \textbf{FeedForward} channel, Hubo-Ach reads it then commands Hubo at the rising edge of the next cycle.}
\label{fig:hubo-ach-feedforward}
\end{figure}


\begin{figure}
\centering

\begin{tikzpicture}[->,>=stealth',shorten >=1pt,auto,node distance=5cm,
  thick,main node/.style={fill=white!20,draw,font=\sffamily\Large\bfseries}]


  \node[main node] (ref) {Reference};
  \node[main node] (filter) [right=3.0cm of ref] {Filter};
  \node[main node] (hubo-ach) [below=1.0cm of filter] {Hubo-Ach};
  \node[main node] (hubo) [right=3.0cm of hubo-ach] {Hubo};




  \path[<->,dashed, every node/.style={font=\sffamily\small}]
    (hubo) edge node [above] {CAN} (hubo-ach);

  \path[->,every node/.style={font=\sffamily\small}]
    (ref) edge node [above] {$\theta_d$} (filter);

  \path[->,every node/.style={font=\sffamily\small}]
    (filter) edge node [left] {$\theta_r$} (hubo-ach);


\end{tikzpicture}
\caption{Desired reference $\theta_d$ being filtered before applied to Hubo via Hubo-Ach.  $\theta_d$ is sent through a filter that reduces the \textit{jerk} on the actuator then the new reference $\theta_r$ is set on the \textbf{FeedForward} channel, Hubo-Ach reads it then commands Hubo at the rising edge of the next cycle.}
\label{fig:hubo-ach-feedforward}
\end{figure}


\begin{figure}
\centering

\begin{tikzpicture}[->,>=stealth',shorten >=1pt,auto,node distance=5cm,
  thick,main node/.style={fill=white!20,draw,font=\sffamily\Large\bfseries}]


  \node[main node] (ref) {Reference};
  \node[main node] (filter) [right=3.0cm of ref] {Filter};
  \node[main node] (hubo-ach) [below=1.0cm of filter] {Hubo-Ach};
  \node[main node] (hubo) [right=3.0cm of hubo-ach] {Hubo};




  \path[<->,dashed, every node/.style={font=\sffamily\small}]
    (hubo) edge node [above] {CAN} (hubo-ach);

  \path[->,every node/.style={font=\sffamily\small}]
    (ref) edge node [above] {$\theta_d$} (filter);

  \path[->,every node/.style={font=\sffamily\small}]
    (hubo-ach) edge node [left] {$\theta_r$} (filter);

  \path[->,every node/.style={font=\sffamily\small}]
    (filter) edge node [right] {$\theta_a$} (hubo-ach);


% look into this and add z^-1

\path [every node/.style={draw,minimum width=3cm, minimum height=5cm]}]
  node (a) at (0,0) {}
  [xshift=7cm]
  node (b) at (0,0) {}
  [xshift=7cm]
  node (c) at (0,0) {};

%\begin{scope}[->,>=latex]
%    \foreach \i in {-2,...,2}{% 
%      \draw[->] ([yshift=\i * 0.8 cm]a.east) -- ([yshift=\i * 0.8 cm]b.west) ;}

%    \foreach \i in {1,2}{% 
%      \draw[->] ([yshift=\i * 0.8 cm]a.east) to [out=50,in=130] ([yshift=\i * 0.8 cm]c.west) ;} 

%    \foreach \i in {-1,-2}{% 
%      \draw[->] ([yshift=\i * 0.8 cm]a.east) to [out=-50,in=-130] ([yshift=\i * 0.8 cm]c.west) ;}
%\end{scope}


\end{tikzpicture}
\caption{Desired reference $\theta_d$ being filtered before applied to Hubo via Hubo-Ach.  $\theta_d$ is sent through a filter that reduces the \textit{jerk} on the actuator then the new reference $\theta_r$ is set on the \textbf{FeedForward} channel, Hubo-Ach reads it then commands Hubo at the rising edge of the next cycle.}
\label{fig:hubo-ach-feedforward}
\end{figure}




As seen in Fig~\ref{fig:singleJointStep} $\theta_c$ tracks $\theta_r$ perfectly. As expected $\theta_a$ lags by a minimum of 1 time step $T$.  
This is the time it takes between sending $\theta_c$ to the actuator over the CAN bus plus the time it takes in receiving the feedback from the encoder of the motor over CAN.
The remainder of the lag is due to the rise time of the actuator.
This is different for each joint.
Because all major joints are high-gain PID the rise-time and overshoot is very small which makes the robot very stiff.
The total lag between commanding the joint on the \textbf{FeedForward} channel and the response of the actuator is:

\begin{equation}
t_{lag} = t_{filter} + t_{rise}
\end{equation}




	\subsection{Joint Space Step Response with Position Filtering}\label{sec:singlejointFilter}
		Giving a step input to a high-gain PID position controlled actuator can cause an over current fault, burn out motor drivers, strip gears due to the \textit{jerk} etc.  
To reduce this effect Hubo-Ach has multiple modes of on-board filtering.
These modes are:
\begin{itemize}
\item Reference Input Filtering
\item Compliance Amplification 
\end{itemize}

This section talks about \textit{reference input filtering} as a method to apply a step input each joint in joint space and limit the jerk.
It is important to note that the obvious answer is to reduce the PID gains to make the robot \textit{more complaint} however the goal of this work is to make a fully functional system that does not require modification of the robot.
In this case the PID gains are set by the motor drivers and that is considered to be a part of the robot.
In future firmware updates of the motor drivers we will have the ability to change PID gains on the fly.

\textit{reference input filtering} uses the history of the previous $\theta_c$ sent to the given actuator.  The current commanded actuator position $\theta_c(N)$ is given by:

\begin{equation}\label{eq:reffiltermode}
\theta_c(N) = \frac{\theta_c(N-1)\cdot\left(L-1\right) + \theta_r(N)}{L}
\end{equation}

Where $L$ is an integer that represents the length of the filter and $L\geq1$.  
If $L=1$ then Equation~\ref{eq:reffiltermode} becomes Equation~\ref{eq:refrefmode}.

\begin{figure}
\centering

\begin{tikzpicture}[->,>=stealth',shorten >=1pt,auto,node distance=5cm,
  thick,main node/.style={fill=white!20,draw,font=\sffamily\Large\bfseries}]


  \node[main node] (ref) {Reference};
  \node[main node] (filter) [right=3.0cm of ref] {Filter};
  \node[main node] (hubo-ach) [below=1.0cm of filter] {Hubo-Ach};
  \node[main node] (hubo) [right=3.0cm of hubo-ach] {Hubo};




  \path[<->,dashed, every node/.style={font=\sffamily\small}]
    (hubo) edge node [above] {CAN} (hubo-ach);

  \path[->,every node/.style={font=\sffamily\small}]
    (ref) edge node [above] {$\theta_d$} (filter);

  \path[->,every node/.style={font=\sffamily\small}]
    (filter) edge node [left] {$\theta_r$} (hubo-ach);


\end{tikzpicture}
\caption{Desired reference $\theta_d$ being filtered before applied to Hubo via Hubo-Ach.  $\theta_d$ is sent through a filter that reduces the \textit{jerk} on the actuator then the new reference $\theta_r$ is set on the \textbf{FeedForward} channel, Hubo-Ach reads it then commands Hubo at the rising edge of the next cycle.}
\label{fig:hubo-ach-feedforward}
\end{figure}





Fig.~\ref{fig:singleJointStepFiltered} shows the commanded reference plotted again the actual reference using the filtered mode defined in Equation~\ref{eq:reffiltermode}.
Fig.~\ref{fig:singleJointStepFilteredLtest} shows the $\theta_r$ plotted against $\theta_c$ and $\theta_a$ for different values of $L$.
It is easy to see that as $L$ increases the $t_{rise}$ also increases and the \textit{jerk} is reduced.


\begin{figure}[thpb]
  \centering
\includegraphics[width=0.8\columnwidth]{./examples/pix/RSP-Zp4-step-filter-real-crop.pdf}
  \caption{The commanded reference plotted against the actual reference recorded  In this plot the commanded reference is automatically filtered by Hubo-Ach.}
%\caption{The commanded reference plotted against the actual reference recorded via Hubo-Ach and ground truth via CAN analyzing utilities.  In this plot the commanded reference is automatically filtered by Hubo-Ach.}
  \label{fig:singleJointStepFiltered}
\end{figure}

\begin{figure}[thpb]
  \centering
\includegraphics[width=0.8\columnwidth]{./pix/tmp.png}
  \caption{$\theta_r$ plotted against $\theta_c$ and $\theta_a$ recorded via Hubo-Ach with values for $L$ ranging from 1 to 200.}
  \label{fig:singleJointStepFilteredLtest}
\end{figure}

This method is a feed-forward method that assumes that the position you set the actuator to is the actual position of the actuator.

	\subsection{Compliance Amplification}\label{sec:singlejointRefComplience}
		Compliance amplification takes advandage of the internal compliance of the joints and amplifies that by feeding back the PID error $\theta_e$.
Like the Equation~\ref{eq:refrefmode} we have no past information about the set reference and we have only the compiliance given by the joints.
If we think about $\theta_e$ and what effects it we can use it to add compliance to our system.
It is important to note that because the Hubo is a high-gain PID position controlled device with an intergral gain $K_i$ set to zero the steady state error of the joint (the PID error $\theta_e$) is proportional to the moment applied to the joint.
If we combine the reference $\theta_r$ and $\theta_e$ multiplied by a compliance gain $K_c$ we are able to add/amplify the compliance to the system.

\begin{equation}
\theta_c(N) = K_c\theta_e(N)+\theta_r(N)
\end{equation}

It is important to note that $K_c \leq 1$ or the system will go unstable.
If $K_c=1$ then we have

\begin{equation}
\theta_c(N) = \theta_a(N)
\end{equation}

	\subsection{Joint Space Step Response with Feedback Filtering}\label{sec:singlejointEnc}
		Feedback filtering allows us to removes the requirement that we know the joint's current position.
Similar to Equation~\ref{eq:reffiltermode} this method sets $\theta_c$ based on a filter length $L$ and the current desired value $\theta_r$.
However instead of assuming that we know all past $\theta_r$ we use the actual position $\theta_a$.


\begin{equation}\label{eq:refencmode}
\theta_c(N) = \frac{\theta_a(N)\cdot\left(L-1\right) + \theta_r(N)}{L}
\end{equation}


\section{Kinematics}\label{sec:hubo-ach-kinimatics}
Kinematic planning is a key focus of the Hubo-Ach controller.
This section provides two published examples of the Hubo-Ach controller being used for inverse kinematics and control.
Section~\ref{sec:valve} shows the work of Lofaro et. al. \cite{lofaroTePRA2013Valve} using Constrained Bi-Directional Rapidly-exploring Random Tree (CBiRRT) to provide a statically stable joint space trajectory allowing the robot to turn a valve.
Section~\ref{sec:6dofik} shows the work of Lofaro et. al. \cite{lofaroTePRA2013HuboAch} using traditional 6 DOF forward and inverse kinematic techniques to provide an analytical IK solution to each of the 6 DOF end effectors.
The additional use of on-line trapezoidal velocity profiling methods in Section~\ref{sec:trap} allow for the creation of a real-time IK controller based in Hubo-Ach.


\subsection{Valve Turning}\label{sec:valve}
	\begin{figure}[thpb]
  \centering
  %\begin{tikzpicture}
    %\clip [rounded corners=1em] (0,0) rectangle coordinate (centerpoint) (5,7.5cm);
%    \node[minimum width=\linewidth,minimum height=174pt,draw=black,rounded corners=1em,fill=bgcolor,draw=black]
%    {};
%    \node[name=img] {
      \includegraphics[width=0.93\columnwidth]{./pix/IMG_9107-small.jpg}
      \includegraphics{./qrcode/qrcode-valve.png}\\
      Video: http://danlofaro.com/phd/valve/
%    };
%    \draw [bgcolor, rounded corners=1em, line width=1em,inner sep=0pt]
%    (img.north west) --
%    (img.north east) --
%    (img.south east) --
%    (img.south west) -- cycle
%    ;
%  \end{tikzpicture}
\caption{Hubo (left) turning a valve via Hubo-Ach alongside Daniel
  M. Lofaro (right).  Valve turning developed in conjunction with
  Dmitry Berenson at WPI for the DARPA Robotics Challenge.}
  \label{fig:valve}
\end{figure}



\section{Six Degree of Freedom Inverse Kinematic Implementation Example}\label{sec:6dofik}
This section shows how we calculate the inverse kinematics (IK) for the Hubo's right arm and how we use that calculation in conjunction with Section~\ref{sec:simpleExamples}.  The result is the ability to command the end effector (EEF)

In order to control the Hubo's upper body manipulators in work space as opposed to joint space both forward and inverse kinematics are required, (FK) and (IK) respectively.
In order to find a proper solution the joint limits, singularities and feasible workspace (no-self collisions) must be accounted for.

The kinematic structure of the right and left arm of the Hubo are identical with the caveat that the work space offset is mirrored over the z-axis.
This means that they have the same Denavit–Hartenberg (DH) parameters

\begin{table}
\centering
\caption{Denavit–Hartenberg for Hubo2+ upper body (arms) in standard format}
\begin{tabular}{|l | c|}
\hline
Link     & Length (m) \\
\hline
\hline
$l_{A1}$ & 0.215 \\
\hline
$l_{A2}$ & 0.179 \\
\hline
$l_{A3}$ & 0.182 \\
\hline
$l_{A4}$ & 0.121 \\
\hline
$l_{E}$ & 0.100 \\

\hline

\end{tabular}\label{table:DHupper}
\end{table}


	\subsection{Froward Kinematics} 
		The transform between joint adjacent joints is represented by the transform:

\begin{equation}\label{eq:dhT}
T_i^{i-1} = \left[ \begin{array}{cccc} 
cos(\theta_i) & -sin(\theta_i)cos(\alpha_i) &  sin(\theta_i)sin(\alpha_i)  &  a_i cos(\theta_i) \\ 
sin(\theta_i) &  cos(\theta_i)cos(\alpha_i) & -cos(\theta_i)sin(\alpha_i)  &  a_i sin(\theta_i) \\
0             &  sin(\alpha_i)              &  cos(\alpha_i)               &  d_i               \\
0             &  0                          &  0                           &  1                 
\end{array} \right]
\end{equation}

Where $\theta_i$ is the 

\begin{figure}[thpb]
  \centering
%  \begin{minipage}{\textwidth}
\includegraphics[width=0.5\columnwidth]{./examples/pix/Sample_Denavit-Hartenberg_Diagram.png}
\caption{Denavit-Hartenberg diagram showing that axis of rotations and displacements to create the transform in Equation~\ref{eq:dhT}.}
Image Credit: \textit{http://en.wikipedia.org/wiki/File:Sample\_Denavit-Hartenberg\_Diagram.png}
%  \end{minipage}
\end{figure}

	\subsection{Inverse Kinematics}\label{sec:ik}
			The next step is to find the inverse kinematic (IK) solution for the righ arm.
Inherently this problem has multiple solutions.
When solving the IK Pieper\cite{peiper1968kinematics} states that a closed-form solution does exist if:
\begin{itemize}
\item Three consecutive joints axes of the manipulator are parallel to one another
\end{itemize}

OR
\begin{itemize}
\item Three consecutive joints intercect at a single point
\end{itemize}

The kinematic structure in Fig~\ref{fig:hubo} and Fig~\ref{fig:IkFkCoordinate} shows that the Hubo2+ platform does have a three joints that intersect the same point in the shoulders and in the hips.
Thus a closed-form solution exists for both arms and both legs.

The transform $T_0^6$ in Equation~\ref{eq:t06} is needed to solve the IK problem for the shoulder.  
It is important to note that $T_0^6$ is in the form of

\begin{equation}
T_0^6 = \left[ \begin{array}{cccc} 
\overline{x_6} & \overline{y_6} & \overline{z_6} & \overline{p_6} \\
0              & 0              & 0              & 1   
\end{array} \right]
\end{equation} 

Where $\overline{x_6}$, $\overline{y_6}$ and $\overline{z_6}$ are $[3x1]$ unit vectors along the principle axes of the end-effectors frame, see Fig.~\ref{fig:IkFkCoordinate}.
Position vector $\overline{p_6}$ describes the hand about joint $A1$ (shoulder).
The arm can be vied in different frames.
If we look at the arm in reference to the end-effector's frame.
The reverse transform is defined as $(T_0^6)`$


\begin{equation}
(T_0^6)` = T_6_0 = (T_ = \left[ \begin{array}{cccc} 
\overline{x_6} & \overline{y_6} & \overline{z_6} & \overline{p_6} \\
0              & 0              & 0              & 1   
\end{array} \right]
\end{equation} 


\section{Verification: Door Opening}\label{sec:hubo-achVerification}
	Section~\ref{sec:timing}, \ref{sec:valve}, and \ref{sec:hubo-ach-kinimatics} verrified the functionality of Hubo-Ach under different circumstances.

\begin{figure}[thpb]
  \centering
      \includegraphics[width=0.69\columnwidth]{./pix/hubo-door.png}\includegraphics[width=0.3\columnwidth]{./qrcode/qrcode-door.png}\\
http://danlofaro.com/phd/door/
      
\caption{Indipendent validation of Hubo-Ach via Zucker et. al.\cite{tepraDoor2013} work in\textit{Continuous Trajectory Optimization for Autonomous Humanoid Door Opening}.}
\label{fig:hubo-ach-door-open}
\end{figure}

Zucker et. al.\cite{tepraDoor2013} independently validates Hubo-Ach through their work in\textit{Continuous Trajectory Optimization for Autonomous Humanoid Door Opening}.
Fig.~\ref{fig:hubo-ach-door-open} shows Zucker's work




\section{Validation: Peer Survey on Hubo-Ach}\label{sec:hubo-achSurvey}
	This section shows the peer survey taken by users of Hubo-Ach.
	Thirteen independent users were surveyed.
	The overwhelming conclusion was that the system is useful, was the unifying algorithmic framework as advertised and helped with development.
	Out of a score from 0-10 on the question \textit{"Would you use Hubo-Ach again the the future when programming Hubo"} received an average of 9.23 (see Table~\ref{table:q6}.
	Table~\ref{table:q1}, \ref{table:q2}, \ref{table:q3}, \ref{table:q4}, \ref{table:q5}, \ref{table:q6}

	\input{survey/q1t.tex}
	\input{survey/q2t.tex}
	\begin{table}
\centering
\caption{Q3: Survey on the Unified Algorithmic Framework for Complex System and Humanoids, Hubo-Ach:}\label{table:q3}
What programming languages do you interface with Hubo-Ach:\\
\small
10 = Often, 0=Never\\
Sample Size = 13\\
\normalsize
%\begin{tabular}{l | l}
\begin{longtable}{|p{9cm} | p{3cm} | }
\hline
Question	&	Average Rating (0-10)	\\	\hline
\hline
\hline
C/C++		& 	9.3 \\
\hline
Python		&	7.69\\
\hline
MATLAB		&	3.15 \\
\hline		
Other		&	1.92 \\
\hline


\end{longtable}
\end{table}

	\input{survey/q4t.tex}
	\begin{table}
\centering
\caption{Q5: Survey on the Unified Algorithmic Framework for Complex System and Humanoids, Hubo-Ach:}\label{table:q5}
Choice of Hubo Software: Given the choice, how likely is it that you would use the following software platforms to implemented your controllers on Hubo.:\\
\small
10 = Very Likely, 0=Unlikely\\
Sample Size = 13\\
\normalsize
%\begin{tabular}{l | l}
\begin{longtable}{|p{9cm} | p{3cm} | }
\hline
Question		&	Average Rating (0-10)	\\	\hline
\hline
\hline
ACES/Conductor		& 	2.69 \\
\hline
Hubo-Ach		&	9.51 \\
\hline
Maestro			& 	5.42 \\
\hline
RAINBOW (Windows)	&	3.46 \\
\hline
RAINBOW (Xenomai)	&	4.00 \\
\hline


\end{longtable}
\end{table}


	\input{survey/q6t.tex}

In conclusion Hubo-Ach is validated as being a useful \textit{unified algorithmic framework for complex systems and humanoid robots} by peers.
Shown to have consistent system performance in Section~\ref{sec:timing}.
It is verified via implementation of full body kinematic examples in Section~\ref{sec:valve} and \ref{sec:6dofik}.








