\subsection{Introduction}



It is common place for a complex electrical mechanical system to have a hybrid controller.  
Essentially different controllers are needed when the system is in different states or doing different tasks.
For the remainder of this document the complex electro mechanical system that I will be talkking about are robots.
The majority of examples given will be in reference to humanoid robotics and the Hubo2+ (KHR-4+) platform.
The Hubo platform is discribed in Section~\ref{sec:hubo}.

A complete autonomous system is an electro mechanical mechanism that is cabiable of making its own decisions.
The combination of high degree of freedom and requirement for autonomy makes the work space broad.
The decisions made are high level and low level.
My work has brought me through research into kinematic planning resulting in my Sparse Reachable Map, visual recognition techniques and over robot system control which resulted in making one of my robots throw the first pitch at a Major League Baseball game and the creating of an viable control system for complex systems.
This work has brought me to my final thesis.
I argue through example and experimentation that multi-process based method of robot control more beneficial for control of electro-mechanical systems using todays technologies.

%High level decisions take care of objectives such as \textit{where to go} and \textit{what to pick up}. 
%These decisions are like our concious decisions.
%Low level decisions take care of balancing, inverse kinimatics etc.
%These decisions can be compaired to our unconscious responses to stimuli, such as balancing, sitting, breating etc.

My work shows through experimentation a viable way of creating a complete autonomous system.  
As proof of viability I show the basic structure of my system Hubo-Ach in Section~\ref{sec:hubo-ach}\cite{lofaroRAM2013}.  
I give step by step examples in Section~\ref{sec:simpleExamples}.
Section~\ref{sec:simulator} shows how we can move from real-time to using a simulated version of the platform in simulation time without having to change the controller.
Section~\ref{sec:task} discribes the experiment which consists of making the robot proform an advanced task that pulls together visual, kinimatic, path planning and other controllers together using this one system.
Section~\ref{sec:results} shows the results of the experiment thus show the viability of the system.
Lastly Section~\ref{sec:conclusion} discusses the results of the work and the future of this system.






%This is demonstrated through experimentation on multiple electro mechanical platforms.
%The primary platform focused on in this document is the Hubo 2 and the Hubo 2+ full-size humanoid robot.
%The Hubo platform is discribed in Section~\ref{sec:hubo}.




\begin{itemize}
\item Creating a system that can monitor its self
\item Avoid self collisions
\item Plan future movements
\item Proform these plans with a higher archical structure such that it stays stable and safe among all else.
\end{itemize}
