
It is common place for a complex electrical mechanical system to have a hybrid controller.  
Different controllers are needed when the system is in different states or doing different tasks.
This is especially true for complete and complex autonomous systems.
I define a complete and complex autonomous system as an electro mechanical mechanism with high degree of freedom (DOF) that is capable of making its own decisions through the use of sensor data processed by its artificial intelligence (AI).
The combination of high DOF and the requirement for autonomy makes the work space broad and controllers complex.
The overarching question becomes; What is the control system structure for a complete and complex autonomous systems with high DOF, a multitude of sensors, AI performing high-level and low-level tasks all while keeping a stable system structure conducive to collaborative work?
My Thesis shows is that a multi-process control structure coupled with the proper timing mechanisms is conducive to answering this question.
It is proven with physical experiments and the creation of Hubo-Ach\cite{lofaroRAM2013}; a fully functional Sim-Time and Real-Time control system for complete and complex autonomous systems.

Through experimentation I prove my control system is a viable way of controlling complete and complex autonomous system and still be conducive to collaborative work.  
A road map of how my research has taken me to my thesis is shown in Section~\ref{sec:roadmap}.
As proof of viability I show the basic structure of my system \textit{Hubo-Ach} in Section~\ref{sec:hubo-ach}.  
I give step by step examples in Section~\ref{sec:simpleExamples}.
Section~\ref{sec:simulator} shows how we can move from real-time to using a simulated version of the platform in simulation time without having to change the controller.
Section~\ref{sec:task} describes the experiment which consists of making the robot preform an advanced task that pulls together visual, kinematic, path planning and other controllers together using this one system.
The techniques used stem from my contributions in Section~\ref{sec:contributions}.
Section~\ref{sec:results} shows the results of the experiment thus show the viability of the system.
Lastly Section~\ref{sec:conclusion} discusses the results of the work and the future of this system.

Before I continue it is important to note that my work has already been validated by my pears because:
\begin{itemize}
\item It was chosen to be the primary control system for the DARPA Robotics Challenge Track-A Team DRC-Hubo, Section~\ref{sec:drc}.
\item It is being used in the NSF-MIRR project\footnote{NSF-MIRR: Major Research Infrastructure Recovery and Reinvestment (MIRR) \#CNS-0960061 sponsored by the the U.S. National Science Foundation (NSF)}.
\item It is currently being used by MIT, WPI, Purdue, Ohio State, Swarthmore College, Georgia Tech, and Drexel University.
\end{itemize}

For the remainder of this document the complete and complex autonomous systems that I will be referring to are robots.
The majority of examples given will be in reference to humanoid robotics and the Hubo2+ (KHR-4+) platform.
The Hubo platform is described in Section~\ref{sec:hubo}.




