The traditional single loop control structure that is used in robot control software such as Orocos\cite{orocos-gadeyne-ijrr2005}, Microsoft Robot Studio\cite{microsoftRobot4437755}, RobotC\cite{robotc}, and LabVIEW\cite{labview} are not suited for high DOF robots.
Due to the nature of these highly redundant complex electrical mechanical system it is common to desire multiple different controllers running in tandem.  
Different controllers are needed when the system is in different states or doing different tasks or performing multiple tasks at the same time.
Combining these controllers is a problem in complex system.
This problem is hard when each controller has different loop rates that are not even multiples of on another, timing requirements (asynchronous vs. synchronous).

Multi-threaded approaches using shared memory allow for compatibility of multiple loop rates such as in Lee et. al.\cite{multi-thread-robot-5602743} with multi-threaded controllers on their humanoids, Rai et. al.\cite{multi-thread-snake-1541141} with multi-threaded controllers on their snake robots and Zheng et. al.\cite{multi-thread-5524083} with multi-threaded controllers on their under water robots.
The multi-threaded approach still has an inherent flaw.
If the parent or one of the other controller threads crashes it is difficult or impossible to restart the controller and still have access the the shared memory.

By using a multi process approach allows controllers to fault and restart with minimal effect on the other controllers.
Typical ways of communicating between different processes is via UDP or TCP/IP such as OpenHRP\cite{openHRP} with their server based control platform for the HRP humanoid robot; Aramaki et. al.\cite{multiPC-arch-1185243} and Lofaro et. al.\cite{lofaroIASTED2011} with their multi-computer based control methods and the popular Robot Operating System (ROS)\cite{ros} by Willow Garage.

Communicating via TCP/IP sockets, such as in OpenHRP and ROS, guarantee that the data is received but it does not guarantee a arrival time.
This means if the checksum fails the message will be sent again increasing the latency of the message.
This does not work well if a real-time control loop is required.
Using UDP does not resend if the checksum fails.
This keeps the latency low and is better for real-time applications such as in the work of Lofaro.
Both UDP and TCP/IP require that the buffer is read before new data is read.
This means that you must read the older data before newer data.
This is called head of line blocking or HOL\cite{ach}.

Newer state information is preferred by robots that work in the physical world over older data.  
Thus it is desired that HOL is eliminated.
This can be done with some forms of inter-process communication (IPC).

OpenHRP and Webots\cite{Webots} are two of the very short list of systems that have simulators that use the same controller as the hardware platforms.
However at this time to the best of our knowledge there is no system that:
\begin{itemize}
\item uses the same controller with the software and hardware systems
\item is inhreently robust by using a multi-process approach
\item uses low-latency methods for controller communications 
\end{itemize}





